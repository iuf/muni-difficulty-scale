\documentclass[a4paper,oneside]{scrartcl}

\usepackage[utf8]{inputenc}
\usepackage[T1]{fontenc}
\usepackage{lmodern}
\usepackage{textcomp} %provides \textdegree command
% \usepackage[nenglish]{babel}

\usepackage{hyperref} % clickable table of contents, index in pdf files

\hypersetup{
    colorlinks,
    citecolor=black,
    filecolor=black,
    linkcolor=black,
    urlcolor=black
}
\newcommand{\unit}[1]{\ensuremath{\, \mathrm{#1}}} % use i.e. \unit{km} in math mode for better units

\title{Difficulty Rating for Muni Disciplines}
\author{International Unicycling Federation}
\date{\today}


\begin{document}

\maketitle

\begin{abstract}
This is the abstract.
\end{abstract}

\newpage

\tableofcontents
\newpage

\section{Motivation}

There are various trail rating systems specific to mountain biking or hiking.
These rating systems are only partially applicable for mountain unicycling (muni)
and cannot be used as an objective base for defining the difficulty of a muni
downhill or cross country track.\\
The goal of this rating system is to provide a more objective and accurate way
of rating difficulties of trail sections as well as complete cross country and
downhill courses.\\
The benefit of a standardized way of rating tracks is that participants of muni
events no longer need to rely on the subjective statements of the event host but
can use the rating scores to assess the difficulty for theirselves.\\
In the future the rating system might also be used to set minimum or maximum
difficulty requirements for world championships or to distinguish between 
easy downhill and downhill.



\section{The M Scale}
\label{sec:difficulty-m-scale}

The M scale describes the technical difficulty of a single trail section. A
section is a part of a trail with a length of 5 to 20 meters.\\
The rating is valid for good conditions (daylight, dry). It is not influenced by how dangerous
the section is.

There are five grades of difficulty (M0 to M5) as described in the following
subsection. The M scale can be used independently to assign technical difficulties.

\subsection{Difficulty Grades}
\label{sec:m-scale-levels}
The following list shows typical characteristics of the different grades. Not all
characteristics have to be satisfied to warrant a respective grade, but at least
two should be fulfilled.

\paragraph{M0}
pavement or solid soil/compact gravel, no obstacles, slope < 20\%,
90\textdegree\ turn within > 2 m and with slope < 10\%

\paragraph{M1}
partly loose soil/gravel,
small obstacles (small stones, flat roots ~ 5 cm),
single 15 cm steps,
slope < 40\%,
90\textdegree\ turn within > 1 m and with slope < 20\%

\paragraph{M2}
loose soil/gravel,
obstacles (stones, roots ~ 10 cm),
single 30 cm steps,
slope < 60\%,
90\textdegree\ turn within > 0.5 m and with slope < 30\%

\paragraph{M3}
loose soil with loose stones (size of few cm),
obstacles (stones, roots ~ 20 cm),
several irregular steps ~ 20 cm each,
drops < 1 m, gaps < 0.5 m,
slope < 80\%,
135\textdegree\ turn within ~ 0.5 m and with slope < 40\%

\paragraph{M4}
very loose/slippery soil with loose stones (size of several cm),
big obstacles (stones, logs ~ 30 cm),
several irregular steps ~ 30 cm each,
drops < 1.5 m, gaps < 1 m,
slope < 100\%,
135\textdegree\ turn within ~ 0.5 m and with slope < 60\%

\paragraph{M5}
very loose/slippery soil with loose stones (size of several cm),
very big obstacles (stones, logs ~ 40 cm),
several irregular steps ~ 40 cm each,
drops > 1.5 m, gaps > 1 m,
slope > 100\%,
135\textdegree\ turn within ~ 0.5 m and with slope < 80\%

\subsection{Grade Refinement}
To refine the scale, grades like M4.5 (a section which has a difficulty between
M4 and M5) are possible.
%TODO: embed images or link official source
A visual indication for all possible grades is given at
\url{https://drive.google.com/?authuser=0#folders/0B6zv9bFd15YXdXFOcFR4Z05IU28}

\section{Scales to Rate Whole Trails}
\label{sec:muni-difficulty-udh-uxc}

These scales are based on the features of whole tracks and are important to
define requirements for downhill or cross country courses at competitions.\\
The UDH and UXC scales use additional criteria on top of the technical difficulty
to describe the overall difficulty of a course, when ridden in one go without stopping.
These criteria are weighted differently for the UXC and UDH scale to account for the
different nature of cross country and downhill.\\
The weighting is achieved through the maximum points assigned to each criteria. The sum
of all maximum points equals the maximum score of the scales.

For each criteria, the calculation is based on linear interpolation on three intervals.
The interval borders are given in the descriptions below and correspond to 0\%,
20\%, 80\% and 100\% of the maximum points.\\
For values below the lowest interval border the minimum number of points is
given, for values above the highest interval border the maximum number of points
is given.\\
The final rating of the track is the sum of all criteria points, rounded to the nearest integer.

\subsection{UDH Scale (Unicycle Downhill Scale)}
\label{sec:udh-scale}

The scale ranges from 0 to 40 points and describes the overall difficulty of a
muni downhill track. Each criteria is described below along with the
interval borders.

\paragraph{Total Length (0-10 points)}
The total length of the trail in km: 0/2/8/10 points given to 0/1/4/8~km

\paragraph{Average Slope (0-10 points)}
The total height difference divided by total length of the trail: 0/2/8/10
points given to 0/10/25/35~\%

\paragraph{Maximum Difficulty (0-10 points)}
The rating of the most difficult section of the trail using the M scale:
0/2/8/10 points given to M 0/2/3.5/5

\paragraph{Average Difficulty (0-10 points)}
The average difficulty rating of the trail using the M scale, 0/2/8/10 points
given to M 0/1/2.5/3.5



\subsection{UXC Scale (Unicycle Cross Country Scale)}
\label{sec:uxc-scale}
The scale ranges from 0 to 40 points and describes the overall difficulty of a
cross country track. Each category is described below along with the
interval borders.

\paragraph{Total Length (0-15 points)}
The total length of the trail: 0/3/12/15 points given to 0/10/25/40~km

\paragraph{Total Ascent (0-7.5 points)} 
The total ascent of the trail: 0/1.5/6/7.5 points given to 0/200/800/1500~m

\paragraph{Maximum Slope Uphill (0-5 points)} 
The slope of the steepest uphill section, 0/1/4/5 points given to 0/10/25/35~\%

\paragraph{Maximum Difficulty (0-5 points)}
The rating of the most difficult section of the trail using the M scale: 0/1/4/5
points given to M 0/1/2.5/4

\paragraph{Average Difficulty (0-7.5 points)} 
The average difficulty rating of the trail using the M scale: 0/1.5/6/7.5 points
given to M 0/0.5/1/2

Excel sheet to calculate the UDH rating: \url{https://docs.google.com/open?id=0B6zv9bFd15YXQlVUV3hQRFRRc3M}
%TODO: to be replaced by an official link


\section{Applying the Scale}

\subsection{Gathering Track Data}
For measuring the length, height difference and total ascent of a trail, the
geometry of the trail has to be determined using global navigation satellite
systems like GPS and/or GLONASS\footnote{See \url{http://glonass-iac.ru/en/}}.
The slope of single sections is ideally measured using an inclinometer.

The data gathering can be done by riding/walking the track, while recording it
with e.g. GPS. The more difficult sections can be repeated afterwards to take
some notes about their M grades.

\subsection{Rating Section Difficulties}
The technical difficulty rating of a single section or of the whole trail has to
follow the definitions of the M scale and should not be decided by a single
person to avoid subjectivity.

%TODO: include 1 or 2 pictures and describe why the shown tracks are how difficult.

\subsection{Calculating UDH or UXC Scores}
The scores are calculated using linear interpolation, so that the following formula can be applied for any value which is
within the range of the minimal criteria measure and the maximum criteria measure.

$s(x) = \frac{p_1 - p_0}{i_1 - i_0} (x - i_0) + p_0$\\
%
$p_0$ represents the lower boundary of the available points. For the first interval this is always 0.\\
$p_1$ represents the upper boundary of the available points.\\
$i_0$ represents the lower boundary of the criteria measure.\\
$i_1$ represents the upper boundary of the criteria measure.\\
$x$ represents the criteria input for which the score should be determined.\\



\subsubsection{UDH Example}

Using the UDH Scale, one could determine the score for a trail with a length of
6~km by putting the values into the formula shown above. 6~km is in between 4 and
8~km, so that 4 represents the lower criteria measure boundary ($i_0$) and 8
represents the upper criteria measure boundary ($i_1$). The lower boundary for
the points is hence 8 and the upper boundary is 10.\\
This results in: $s(6) = \frac{10 - 8}{8 - 4} (6 - 4) + 8 = 9$\\

Let us further assume that the track has an average slope of 16\% (corresponding to about 9 degrees).
Here, the interpolation is done on the interval from $i_0 = 10\%$ to $i_1 = 25\%$ yielding points 
between $p_0 = 2$ and $p_1 = 8$. \\
The equation thus reads: $s(16) = \frac{8 - 2}{25 - 10} (16 - 10) + 2 = 4.4$\\

After long and fierce debates the local muni riders decided that the
most difficult section of the track warrants a rating of M3.5. 
In this case the points for maximum difficulty are obtained easily: 
M3.5 corresponds exactly to the interval border 8 and no further
interpolation is necessary. \\

The track also includes lots of easy sections resulting in an average difficulty
of M1.5. The points for this criteria are calculated by interpolating between M1 and M2.5
(boundaries for 2 and 8 points, respectively): 
$s(1.5) = \frac{8 - 2}{2.5 - 1} (1.5 - 1) + 2 = 4$\\

The UDH rating of the track consists of the sum of the points for the individual criteria, 
rounded to the nearest integer:\\
$\mathrm{UDH} = 9 + 4.4 + 8 + 4 = 25.4 \approx 25$\\
The result is a UDH rating of 25 out of 40 points. For convenience these calculations 
can be automatically done using the online calculator: %TODO: link to online tool

\subsubsection{UXC Example}
For the UXC scale the same algorithms (i.e. linear interpolation) as for the UDH scale are used. However, different criteria and point distributions contribute
to the final rating. The example will present the calculations for a long and easy course.\\

{\parindent0pt
Total Length: 42~km\\
42~km is longer than 40~km so the full 15 points are given. \\

Total Ascent: 590~m\\
Interpolation on the interval 200--800~m / 1.5--6 points:\\
$s(590) = \frac{6 - 1.5}{800 - 200} (590 - 200) + 1.5 = 4.4$\\

Maximum Slope Uphill:  8\% \\
Interpolation on the interval 0--10~\% / 0--1 points:\\
$s(7) = \frac{1 - 0}{10 - 0} (8 - 0) + 0 = 0.8$\\

Maximum Difficulty: M1.5\\
Interpolation on the interval M1--2.5 / 1--4 points:\\
$s(7) = \frac{4 - 1}{2.5 - 1} (1.5 - 1) + 1 = 2$\\

Average Difficulty: M0.5\\
M0.5 corresponds exactly to 1.5 points.\\

Final UXC rating:\\
$\mathrm{UXC} = 15 + 4.4 + 0.8 + 2 + 1.5 = 23.7 \approx 24$\\
}

\subsection{Publishing Scores}
Hosts of muni events are encouraged to publish the scores for the event tracks
in advance so that participants know what kind of track expects them.\\ The
score should be published together with a reference to this document or with a
short description of the rating system.

\section{Feedback and Improvements}
The rating system described in this publication is an experimental version which
requires intensive testing. Ideas for improvements or other feedback is welcome
at any time and can be sent to the IUF.
%TODO: proper contact details? no email in clear text to avoid spam

\end{document}
